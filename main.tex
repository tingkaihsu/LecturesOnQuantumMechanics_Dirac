\documentclass[12pt]{article}
\usepackage{braket}
\usepackage{physics}
\usepackage{graphicx}
\usepackage{times}
\usepackage[export]{adjustbox}
\usepackage{listings}
\usepackage{mathcomp}
\usepackage{hyperref}
\usepackage{bm,amsmath}
\usepackage{float}
\usepackage{indentfirst}
\usepackage{bigints}
\usepackage{listings}
\usepackage{color}
\hypersetup{
colorlinks=true,
linkcolor=blue,
filecolor=magenta,
urlcolor=cyan,
pdftitle={Overleaf Example},
pdfpagemode=FullScreen,
}
\definecolor{dkgreen}{rgb}{0,0.6,0}
\definecolor{gray}{rgb}{0.5,0.5,0.5}
\definecolor{mauve}{rgb}{0.58,0,0.82}
\lstset{frame=tb,
language=Python,
aboveskip=3mm,
belowskip=3mm,
stepnumber = 1,
showstringspaces=false,
columns=flexible,
basicstyle={\small\ttfamily},
numbers=left,
numberstyle=\color{gray},
keywordstyle=\color{blue},
commentstyle=\color{dkgreen},
stringstyle=\color{mauve},
breaklines=true,
breakatwhitespace=true,
tabsize=3
}
\numberwithin{equation}{section}

\title{Notes of Lectures on Quantum Mechanics by P.M.Dirac}
\author{Ting-Kai Hsu}
\date{\today}

\begin{document}
\maketitle
\tableofcontents
\section{Transformation That Doesn't Change State}
If we start with given initial variables and solve for the equations of motion, whose solutions should contain arbitrary functions of time $v_a(t)$ as we discuss above.
The initial variables we need would $q\text{ and }p$ but not the initial value of arbitrary functions $v_a(t)$, and this is quite important because \textit{initial physical state} of the system is uniquely determined by $q$ and $p$ but not $v$, which is consistent with the system without constraints.
\\\indent As time evolves, the initial state must determine the state at later time. 
However, $q$ and $p$ would not be uniquely determined by the initial state due to the arbitrary functions in solutions of equations of motion.
That means \textit{the state doesn't uniquely determine $q$ and $p$ but a given set of $q$ and $p$ would uniquely determine a state.}
There must be several choices of $q$ and $p$ which correspond to the same physical state.
To explicitly show this, consider a general dynamical variable $g$ with initial value $g_0$ and a short time interval $\delta t$.
The value of this dynamical variable at time $\delta t$ would be given by the equations of motion,
\[g(\delta t) = g_0 + \dot{g}\delta t\]
\[ = g_0 + \left[g, H_{T}\right]\delta t\]
\begin{equation}
    \approx g_0\delta t\left\{[g, H'] + v_a[g, \phi_a]\right\}
\end{equation}
The coefficients $v$ are completely arbitrary thus we can consider different functions of time $v'$, which therefore gives a different $g(\delta t)$.
Define the difference between different coefficients,
\begin{equation}
    \Delta g(\delta t) = \epsilon_a\left[g, \phi_a\right]
\end{equation}\label{1.2}
where
\begin{equation}
    \epsilon_a = \delta t(v_a - v'_a)
\end{equation}
which is a small arbitrary parameter, small because of short time interval $\delta t$ and arbitrary because of $v$ and $v'$.
Consider changing all the Hamiltonian variables (like coordinates and canonical momentum) with the rule eq(\hyperref[1.2]{1.2}) yet would describe exactly the same state.
We come to the conclusion that the $\phi_a$ which appear in the theory as primary, first-class constraints, have the meaning:
\textit{as generating functions of infinitesimal contact transformations, they lead to changes in coordinates and canonical momenta that don't affect the physical state}.
\\\indent  To push the theory further by considering successive contact infinitesimal transformations.
First by $\epsilon_a\phi_a$ and then $\gamma_{a'}\phi_{a'}$, thus we have,
\begin{equation}
    \begin{split}
        \Delta g = \epsilon_a\left[g, \phi_a\right]\\
        \Delta' g = \gamma_{a'}\left[g+\Delta g, \phi_{a'}\right]
    \end{split}
\end{equation}

Write them altogether,
\begin{equation}
    g''_1(\delta t) = g(\delta t) + \epsilon_a\left[g, \phi_a\right] + \gamma_{a'}\left[g+\epsilon_a\left[g, \phi_a\right], \phi_{a'}\right]
\end{equation}

However, if the transformations are applied in the reverse order, 
\begin{equation}
    g''_2(\delta t) = g(\delta t) + \gamma_{a'}\left[g, \phi_{a'}\right] + \epsilon_{a}\left[g+\gamma_{a'}\left[g, \phi_{a'}\right], \phi_a\right]
\end{equation}

These two would correspond to different change in $q$ and $p$, yet both of them wouldn't change the physical state.
The difference between these two quantity would yield,
\[\Delta_{12}g\equiv g''_1 - g''_2 \approx \epsilon_a\gamma_{a'}\left\{\left[\left[g, \phi_a\right], \phi_{a'}\right] - \left[\left[g, \phi_{a'}\right], \phi_a\right]\right\}\]
Further simplify this by Jacobi identity, 
\[\left[\left[g, \phi_a\right], \phi_{a'}\right] + \left[\left[\phi_a, \phi_{a'}\right], g\right] + \left[\left[\phi_{a'}, g\right], \phi_a\right] = 0 \]

Then the equation becomes,
\begin{equation}
    \Delta_{12}g = \epsilon_a\gamma_{a'}\left[g, \left[\phi_a, \phi_{a'}\right]\right]
\end{equation}

Therefore, we immediately see that $\Delta_{12}g$ represents a change in Hamiltonian variables yet does not involve any change in the physical state.
We use,
\begin{equation}
    \left[\phi_a, \phi_{a'}\right]
\end{equation}
as a generating function of infinitesimal contact transformation that causes no harm to the physical state.
\\\indent By the theorem previously proved, the Poisson brackets of first-class constraints would be first-class, too.
\\\indent The final result is that those transformations of dynamical variables which do not change physical states are infinitesimal contact transformations in which the generating functions is a primary first-class constraint or possibly a secondary first-class constraint.
\section{The Problem of Quantization}
We were led to the idea that there are certain changes in the $p$'s and $q$'s that don't correspond to a change of state, and which have as generators first-class secondary constraints.
This implies a more general equation of motion that allows variation of a dynamical variable $g$ with the time contains \textit{not only} any variation given by equation (1.1) $\dot{g} = [g, H_{T}]$, \textit{but also} any variation which does not correspond to a change of current state.
\begin{equation}
    \dot{g} = \left[g, H_{E}\right]
\end{equation}
where $H_{E}$ stands for the extended Hamiltonian, consisting of the previous total Hamiltonian $H_{T}$, plus all generators that don't change the current state, with arbitrary coefficients\footnote{Recall that we mentioned in the last section that the arbitrariness of a system is determined by the number of generators that don't change the system.}:
\begin{equation}
    H_{E} = H_{T} + v'_{a'}\phi_{a'}
\end{equation}
Those additional generators $\phi_{a'}$, which haven't been included in $H_{T}$ are the first-class secondary constraints.

The general Hamiltonian theory here applies to a finite number of degrees of freedom.
We could easily extend $N$ into infinity with the original indices denoting the degrees of freedom is $n = 1,\cdots, N$.
Furthermore, the suffix could be taken from discrete to continuous, that is, we have $q_{n}$'s and $p_{n}$'s variables becomes $q_{x}$ and $p_{x}$, where $x$ is the suffix which could be any values in a continuous range.
In continuous case, the sums over $n$ in the previous work would become integrals.

However,
\end{document}